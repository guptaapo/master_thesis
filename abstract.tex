%% ----------------------------------------------------------------------------

\newpage
\vspace{3cm}

\chapter*{Abstract}

``The next crash risk is hiding in plain sight-Sometimes the ticking time bomb is in corners of the system that seem dull and safe''- says Gillian Tett. Sometimes market shocks and bubble occurs because of risky bets like Long-Term Capital Management (LTCM) fund crisis (1987), Portfolio insurance debacle (2007), Quant crisis, Lehman brother bankruptcy in 2008 or the recent house bubble. This situation may be further aggravated in the next decades by the increase in the financialization through Exchange traded funds (ETFs), speed and automation through algorithm trading, and public debt \cite{a5}. The adverse effect of the crisis has lead to a massive exodus of most of the clients, devaluation of the assets, instability in the market, break down in pricing mechanism of the stock market and so on so forth. Today, western banks are well capitalized and regulated by Basel terms. And the economy is flushed with the cash and seems to be calm which reflects a sense of security. But this calmness will not only lead to the danger of risky bets but also for the pearls of safe assets too. 

Even though ETFs led to the Great Crash, but still they are considered to be safe investments. This sector has recently exploded in size with more than 4 billion in asset under management (AUM). Passive and quantitative investors have covered more than 60\% of the AUM, which was under 30\% a decade ago. De planta \cite{a3} has concluded, ``If the majority of us embrace them, index-trackers threaten to sabotage the entire economic system''. This inclination towards passive investment is due to higher fees charged by the active managers and they have underperformed in past decades. Therefore, in order to maintain balance in the economy and outperform the passive investments in the market, active fund managers fight back against ‘Darwinian cull’. Active managers turn to strategies that are difficult to replicate in a passive format. 

This motivated us to develop a new model to help active managers to outperform in the market. We developed a Statistical Agent-Based model (SABM) to solve the problem at hand. The strength of SABM lies in its capability to shift the regime from microscopic to macroscopic level and thus, resolving the complex economic problems. The strategy is to benefit from the tail risk which emanates from crowding which is not adequately priced. In this thesis, we describe the steps for developing a SABM model and formulate the calibration of the model as an optimization problem. The ease of calibrating a model in case of SABM provides an advantage over the use of ABMs. 

Finally, the model is evaluated using historical S\&P 500 index data. We evaluate our model using random trading strategies and linear regression. Various experiments with different window lengths for calibration are used. The preliminary results encourage the prediction and, also conclude that the model provides relevant information. The research findings of this thesis can be used by the active managers in the industry to scientifically justify their business model. It will also help the agents to view the effect of market factors globally.