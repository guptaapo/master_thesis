%% -----------------------------------------------------------------------
%% -----------------------------------------------------------------------
\chapter{Statistical Analysis}
In the previous chapter, we explained the method to calibrate the model and also, the method to obtain the market parameters using maximum log-likelihood estimate (MLE). Once the model is calibrated, we can predict the returns and variances using the same model. The next step is to do statistical analysis on the predictions to measure the performance of our model.

\section{Trading strategy}
Once we have predicted daily returns and standard deviation for the time period under observation, we can design a trading strategy based on these predictions. For example, the strategy can be as follows: 

\par Using historical data from 1990-01-01 to 1999-12-31, we can predict expected values for 2000-01-01. Let the expected return on 2000-01-01 be 0.3\%. The trading strategy can be \textit{to buy}, if the return is higher than a threshold, \textit{to short}, if the predicted return is lower than negative of threshold and otherwise, do nothing. If threshold chosen is 0.1\% in this strategy, we will make buy on 2000-01-01. We can generate the decision signal for all required days.\newline

For analysis, we can try different thresholds or even different strategies. Another option is to check with predicted return divided by standard deviation value, else we just buy without short, or do short without buy.
Using the trading strategy, the daily returns for the whole period of observation can be obtained. We observe the daily return for the trading strategy and simultaneously count the number of trading days ($NT$), number of buy days ($NB$) and number of short days ($NS$) in the observation period. Let $r_{real}(t)$ be daily real returns, then the daily return of my trading strategy, $r(t)$, can be calculated as follow:

\begin{equation} \label{eqn:ret}
r(t) = 
\begin{cases}
  r_{real}(t), & \text{for buy days} \\
  -r_{real}(t),& \text{for short days} \\
  0,  & \text{else}
\end{cases}
\end{equation}

\section{Evaluation of trading strategy} 
After recording the returns for our trading strategy, the next step is to evaluate the performance of the returns. We did the following tests on our returns:

\begin{enumerate}
\item Comparison with random strategies

\item Linear Regression
\end{enumerate}

\subsection{Comparison with random strategies} \label{ssec:random}
This test is used to confirm whether the model used and the trading strategy designed are able to add information while predicting. This is done by recording daily returns for randomly generated strategies. \par

There can be many different ways to generate random signals. One of the common approaches is to select a time period ($T$) randomly within the observed period. Within the selected time period, we count the number of the buy days ($NB$) and the short days ($NS$). We calculate the total number of trading days ($NT$) as the sum of the buy and short days. The next step is to randomly select $NT$ days from the time period, $T$. From these $NT$ trading days, $NB$ days are selected randomly and labeled as buy days, while remaining $NT-NB$ days are labeled as short days. This can be seen as generating a random signal with the constraint of having the same number of buy and short days as in our trading strategy. Finally, we use this random signal with buy and short days to generate returns using equation \ref{eqn:ret}. The next step is to compare these daily returns against returns obtained from the trading strategy from our model. We compare the performance on the following indicators:

\begin{itemize}
\item Compound annual growth rate(CAGR): the mean annual growth rate of an investment over a specified period of time longer than one year \cite{a10}.

\item Sharpe ratio: It is the excess return divided by the volatility.

\item Maximum drawdown: It is calculated as the difference between the peak and trough divided by the peak.

\end{itemize}

The returns are generated for a huge number of random signals and the performances are compared. We expect the performance of our strategy to be better than at least 90\% of the performances obtained from random strategies. If this condition is satisfied, it means that our trading strategy works not by chance, and it provides some information in the model.


\subsection{Linear Regression}\label{ssec:linear}
Once random strategy test is done, we used another option to validate the performance of the model. Using the recorded returns, $r(t)$, we run a linear regression model with \textit{FAMA-FRENCH 3 factors model}. The linear regression model is formulated as:

\begin{equation}
r(t) = \alpha + \beta_{HML} \ HML(t) + \beta_{SMB} \ SMB(t) \ ,
\end{equation}
where $\alpha$, $\beta_{HML}$ and $\beta_{SMB}$ are coefficients, $HML(t)$ is high-minus-low (HML) factor and $SMB(t)$ is small-minus-big (SMB) factor. HML is the factor representing the effect of value of company, while SMB is the factor referred as the "size effect". \par

If the regression model gives a significant intercept ($\alpha$), it means that something in the $r(t)$ cannot be explained by the 3 or 4 factors. If that is the case, it means there is some information in the model. In order to run the linear regression, we made use of the statistics model package in python.
