%% ----------------------------------------------------------------------------
%% ----------------------------------------------------------------------------

\chapter{Conclusion}
We have constructed a Statistical Agent-based model which can be populated by artificial agents i.e. fundamentalists and chartists. This model is developed to understand the complexity of the financial market. It is a new way of solving the macro-level economic problems. Moreover, it tries to overcome the drawback of  Agent-based models. The ABMs are considered to be complex and have non-linear chaotic behavior. Also, they only take micro-level factors into consideration and is quite difficult to calibrate for high dimensional problems.

This thesis shows that SABM is an efficient way to deal with multiple factors due to its layering structure. The layering structure allows us to add multiple agents in the first layer and thus, does not add any complexity to the calibration. The calibration is done in next layer using the real returns for a given in-sample window. The calibration is formulated as an optimization problem using Maximum Likelihood Estimation, which provides the optimal values of the five market parameters. In this thesis, we also observe the advantage of re-calibrating the model to obtain best future predictions. These market parameters are further used for predicting the returns in each out-of-sample window. We generate a trading signal using the predicted returns. This is done by using two different trading strategies. Further, these signals are used to compute trading returns from real returns. 

In order to validate our model, we compare the performance of our model with random strategies on various indicators. We computed various results using different window lengths in which Case III outperforms the random strategies in terms of P\&L, Sharpe ratio, Maximum drawdown, and CAGR. Also, the results of linear regression for Case III are better than other cases. It shows a positive $\alpha$ and significant coefficients for market factors. 

From these results, we can conclude that the preliminary results are encouraging the prediction. The results can be further improved by performing more experiments. In these experiments, we can select different window lengths, step sizes, and the trading strategies, and do a detailed statistical analysis on the performance of our model.