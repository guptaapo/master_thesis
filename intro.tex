%% ----------------------------------------------------------------------------
%% ----------------------------------------------------------------------------


\chapter{Introduction}
Passive funds have made a big chunk of overall assets under management in Asian and US equities and a smaller - but rising - proportion in areas such as US bonds. Over time, active management is seen to be underperforming across all major geographies, in both developed and emerging. Higher fees and trading friction in active management, increment costs for investors. As long as corporate governance improves to developed-market standards, passive investment is likely to grow overseas at a rate similar to US market.

Moreover, the underperformance of active fund managers tends to push institutional investors towards 'passive' management of their assets. Indeed, the department for communities and local government in the UK has recently suggested that almost £85 billion of defined benefit pension funds could be moved from active to passive management \cite{a2}. The rationale for the suggestion is that the average returns from active management may not justify its higher costs \cite{a2}. In the US, the majority of people prefer to invest in the pension funds, which further invest in equity (which is indexed and around 11 trillion USD). In addition, mechanically investing in an index that is 100\% invested in the equity market requires the pensioner to take on far more risk that he most likely wants. 

This suggests that the two massive bear markets over the last decade have made investors lost something far more valuable than money – the time that was needed to reach their retirement goals. From a regulatory perspective, it is therefore interesting to understand how the trend towards indexation will impact the social welfare \cite{a1}.

A crowded trade generally grows around what started as a good idea such as portfolio insurance, which most of the pension funds used. Due to herding effect, copycats learn to enter into space and capital flows into the strategy. This leads to even more success for those already there, until, at one point, the pendulum swings and the music stops. Indexing imposes a non-linearity that drives the most overpriced stocks to become even more overpriced. That is precisely why the lofty valuations on the FAANGs just keep getting loftier. The first hint of trouble causes cash inflows to dry up and buying to stop. The redemption by drawdown-sensitive (active) investors cause instantaneous selling and leads to the bubble in the market.

Moreover, passive indices have both concentration risk and unwelcome biases and therefore, leading to the systemic risk. The main concern is that the ascent of the passive investments will make the market more chaotic, unpredictable and brittle. Many investors claim that the shift from active to passive is mostly in Exchange Traded Funds (ETFs). The increased ownership of ETFs is detracting from the stock market efficiency. Some fund managers and analysts can detect the warning sign of bubbles in passive investment. A bubble is defined as the period of the unsustainable growth when the price of an asset follows a faster-than the exponential power law growth i.e hyperbolic growth. This growth is often accompanied by the log-periodic oscillations over and above passive tide.

Renaud de Planta \cite{a3}, an active investment manager, the chairman of Pictet Asset Management, states that - ``A cure-all. This is what passive investing represents to its growing band of proponents. Equity tracker funds, we’re told, will rid the financial market of toxic elements and restore it to full health. At first glance, it’s a persuasive argument. Poorly performing and expensive active managers have lingered in the system for too long, eroding returns for investors. Yet on deeper reflection, index-tracking products are no miracle remedy. They are more like antibiotics: valuable when deployed in moderation, but likely to do more harm than good, should their use become widespread.''

He claims that passive investing erodes competitive forces because companies in the same sector end up with the same investor base, which is probably where he is on the strongest ground. But he also argues that if passive funds monopolized investment flows, pricing mechanisms in the stock market would break down. He suggested that the price of a stock would no longer reflect a company's actual performance because their shares would be bought simply as a result of their inclusion in an index.

In order to understand the complexity in finance and economy, one should step back from the traditional approaches such as - expected utility maximization or maintain equilibrium in the market and should try an innovative approach to avoid the bubble in the financial market. As Sherlock Holmes solves mysteries, we should look the financial market from the alternative viewpoint \cite{a4}. i.e.  ``Once you eliminate the impossible, whatever remains, no matter how improbable, must be the truth.''

In order to mitigate this risk, active management should efficiently allocate the capital within the market. The active managers are turning away from the traditional strategy of comparing their own performance relative to the equity benchmark, instead, they focus on providing the absolute results to the investor in any market condition. Another defensive strategy opted is by taking more aggressive bets on the active shares and increasing its weight in the portfolio and providing alpha. Instead of using standard accounting data, they can use quantitative strategies based on new data and apply the significant computational ability to outperform the passive funds. In this research paper, a strategy has been formulated to enable active managers to benefit from the tail risk, emanating from crowding which is not adequately priced and can help them to outperform.

\section{Our Approach}
In order to solve the problem statement, at first, we try to develop a model which simulates the trading behavior of the individuals in the market. To model the behavior, we can make use of various computational models like Compartmental models, Agent-based models (ABM), Decision-Analytic models etc. Although, these methods are very efficient but for our research which is focused more on the behavior of the system over time, Agent-based models look to be more relevant.

ABM is used to study the large-scale phenomena arising from micro-interaction \cite{abm_ref}. The model considers the heterogeneity of the agents by doing the parameterization using two variables, which are different for each agent. In today's world, the artificial market can help us to understand the impact of market rules on the behavior of the market makers and traders. It simulates the behavior of the system over time. It uses the bottom-up or individual level approach i.e. how the behavior of individuals can affect the overall behavior of the system. It shows how the virtual person might behave in the simulated community. Moreover, these models are low cost, flexible and provide the natural description. The agents in these models make trading decisions based on the history of changing directions in prices. In general, the model has limited memory of length ($m$), which is the same for all agents. Each agent is provided with the same number of trading strategies ($s$), but in general, different agents may have different trading strategies. The decisions are made by learning from the historical performance of their trading strategies. They analyze the performance of trading strategies and choose the best strategy to make future decisions. 

Although, ABM has a lot of advantages but due to its non-linear structure and stochasticity in the individual behavior, it has complex interactive networks. Moreover, these models have certain limitations. The results obtained using these models are uncertain. The results depend heavily on the input values and the internal structure of the model. In addition, it is difficult to correlate the output with the input ingredients \cite{yukalov}. It follows the parallel world universe and does not fit in solving the micro-macro problems. Moreover, due to its complex dynamics and non-linear chaotic behavior, it is quite difficult to calibrate and validate the model \cite{ising}. Other researchers also tried to apply maximum likelihood estimation to do the calibration, but ABMs have the issue of dimensionality and ill-conditioning i.e. small error gets accumulated into a large error while using the calibration methods.

This diagnostic has given the opportunity to others to come up with various options to remove these drawbacks. Windrum et al. \cite{windrum} review the calibration and validation problem of ABM in economics and classify the calibration approach into three categories:- (i) the indirect calibration approach, (ii) the Werker-Brenner approach, and (iii) the history-friendly.

Instead of using above methods, we decided to develop a new model, the Statistical Agent-based model (SABM) to overcome drawbacks of ABM. SABM is a new way to add things up. It is a shift from micro to macroscopic level to resolve the complex problem and understand the various players in the market. The model is built on the concept of multiple layering i.e we add the two groups, fundamentalists, and chartists, which have different statistics and distribution curve. This is called inner layering. This layer provides the performance and decision of both agents. The next layer predicts the macroscopic variable by doing calibration and re-calibration of the model. This enables us to reverse engineer and analyze the problem in the real universe. Further, SABM provides a reason for stylized facts in financial time series, such as excess volatility, temporary bubbles and trend following, sudden crashes and fat tails in the returns distribution \cite{ising}.

\section{Thesis Organization}
In chapter 2, implementation and use of SABM model is well explained. Next, using historical data and back-testing, we calculate decisions (to hold or not to hold) and compute performance i.e. annual return and volatility, for the model created. Using these computations, we try to calibrate our model to mimic the market behavior.  The details of methods used for calibration are explained in chapter 3. After calibration, we are able to predict the expected returns and volatility for the future dates. In chapter 4, we analyze the future data to identify the possibility of a crash or financial bubbles in the market. The results obtained are shown in chapter 5. The thesis ends with the chapter 6 containing conclusion and discussion. 

