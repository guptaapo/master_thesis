\usepackage{mathtools}
\usepackage{xspace}
\usepackage{etoolbox}
\preto\subequations{\ifhmode\unskip\fi}
%-- TODO: MOVE OUTSIDE THIS FILE
% Latin abbreviations 
\makeatletter
\newcommand\ensuresingleperiod{\@ifnextchar.{}{.\@\xspace}}
\makeatother

\newcommand\foreign[1]{\emph{#1}}
\newcommand\eg{\foreign{e.g.}\xspace}
\newcommand\ie{\foreign{i.e.}\xspace}
\newcommand\vs{\foreign{vs.}\xspace} 
\newcommand\cf{\foreign{cf.}\xspace} 
\newcommand\etc{\foreign{etc}\ensuresingleperiod}
\newcommand\apriori{\textit{a priori}\xspace}
\newcommand\aposteriori{\textit{a posteriori}\xspace}
\newcommand\invivo{\textit{in vivo}\xspace}
\newcommand\insilico{\textit{in silico}\xspace}
\newcommand\invitro{\textit{in vitro}\xspace}
\newcommand\denovo{\textit{de novo}\xspace}
%-- TODO: MOVE OUTSIDE THIS FILE


% Number Fields
% Reals
\newcommand*\R{\mathbb{R}}

% \eqdef, \defeq (equal by definition) : :=, =:
\newcommand{\defeq}{\coloneqq}
\newcommand{\eqdef}{\eqqcolon}

% \indicator : indicator of a set
\newcommand{\indicator}{\ensuremath{\mathbf{1}}}


% \diff : differential operator
\newcommand{\diff}{\mathop{\mathrm{{}d}}\mathopen{}}

% \set, \set*, \set[\big] etc. (see mathtools): typesetting sets
%\DeclarePairedDelimiter{\set}{\lbrace}{\rbrace}
%\DeclarePairedDelimiter{\paren}{\lparen}{\rparen}
% \DeclarePairedDelimiter{\bars}{\lvert}{\rvert}
% \DeclarePairedDelimiter{\dbars}{\lVert}{\rVert}
% \makeatletter
% \newcommand\@normStar[1]{\dbars*{#1}}
% \newcommand\@normNoStar[2][]{\dbars[#1]{#2}}
% \newcommand\norm{\@ifstar\@normStar\@normNoStar}
% \makeatother




% This new command ensure right spacing
\newcommand{\interval}[4]{\mathopen{#1}#2 \mathclose{}\mathpunct{}, #3\mathclose{#4}} 
\newcommand{\intervalcc}[2]{\interval{[}{#1}{#2}{]}} % interval close close : [ , ]
\newcommand{\intervaloc}[2]{\interval{]}{#1}{#2}{]}} % interval open close : ] , ]
\newcommand{\intervalco}[2]{\interval{[}{#1}{#2}{[}} % interval close open : [ , [
\newcommand{\intervaloo}[2]{\interval{]}{#1}{#2}{[}} % interval open open : ] , [

% \smallo, \bigO : small/big o/O Landau notation for asymptotic equivalents
% \mathopen removes a spurious space induced by the opening of the parenthesis

\newcommand{\littleo}{o\mathopen{}\paren}
\newcommand{\bigO}{O\mathopen{}\paren}



% \abs, \card and \norm: absolute value, cardinal and norm
\DeclarePairedDelimiter{\abs}{\lvert}{\rvert}
\DeclarePairedDelimiter{\card}{\lvert}{\rvert}
\DeclarePairedDelimiter{\norm}{\lVert}{\rVert}

% \norm

% \vec, \mat : typesetting vectors and matrices
\renewcommand{\vec}[1]{\ensuremath{\boldsymbol{#1}}}
\newcommand{\mat}[1]{\ensuremath{\boldsymbol{\MakeUppercase #1}}}


% \transpose, adjoint : transpose, adjoint of a matrix/operator
\newcommand{\transpose}[1]{\ensuremath{#1^{T}}\xspace}
\newcommand{\adjoint}[1]{\ensuremath{\transpose{#1}}\xspace}

% \crossprod : typesetting cross product
\newcommand\crossprod{\mathbin{\times}}



% \funop : define symbols representing functions, for correctly
% handling spacing
%\newcommand\funop[1]{\mathop{{}#1}}
\newcommand\funop[1]{\mathop{{}#1}}


%-- Paper Specific Macros

% Operators and operations
\DeclareMathOperator{\dgrad}{D}
\DeclareMathOperator{\PSFop}{K}

\newcommand\ConvSymbol{*}
\newcommand\conv{\mathbin{\ConvSymbol}}

\newcommand\LHS{\mathrm{L.H.S.}}
\newcommand\RHS{\mathrm{R.H.S.}}


% Variables

% \Tu, \Tv: Tangent vectors to a surface 
%\newcommand*\Tu{\vec{T}_{\!u}}
%\newcommand*\Tv{\vec{T}_{\!v}}
\newcommand*\x{\ensuremath{\vec{x}}\xspace}
\newcommand*\dx{\diff \x}

\newcommand*\p{\ensuremath{\vec{p}}\xspace}


\newcommand*\y{\ensuremath{\vec{y}}\xspace}
\newcommand*\dy{\diff \y}


\newcommand*\dt{\diff t}

\newcommand*\dl{\diff l}

\newcommand*\du{\diff u}
\newcommand*\dv{\diff v}


\newcommand*\PSF{\ensuremath{K}\xspace}
\newcommand\fPSF{\funop{\PSF}\paren}



\newcommand*\Curve{\ensuremath{\mathcal{C}}\xspace}
\newcommand*\Surf{\ensuremath{\mathcal{S}}\xspace}
\newcommand*\Manifold{\ensuremath{\mathcal{M}}\xspace}
\newcommand*\Manifoldt{\ensuremath{\mathcal{M}_{t}}\xspace}
\newcommand*\ParamDomain{\ensuremath{\mathcal{D}}\xspace}

\newcommand*\PhysDomain{\ensuremath{\Omega}\xspace}

\newcommand*\Pixel{\ensuremath{\mathcal{P}}\xspace}
\newcommand*\TInterval{\ensuremath{\intervalcc{0}{T}}\xspace}

\newcommand*\PhotonFlux{\ensuremath{\Phi}\xspace}
\newcommand*\fPhotonFlux{\funop{\PhotonFlux}\paren}

\newcommand\PhotonFluxE[1]{\ensuremath{\Phi_{\mathrm{E}_{#1}}}\xspace}

\newcommand\Photo{\ensuremath{\phi}\xspace}
%\newcommand\fPhoto{\funop{\Photo}\paren}

\providecommand\given{}
%\DeclarePairedDelimiterXPP{\fPhoto}[1]{\funop{\Photo}}(){}{\renewcommand\given{\nonscript\:\delimsize\vert\nonscript\:\mathopen{}}#1}
\DeclarePairedDelimiterXPP{\fPhoto}[1]{\funop{\Photo}}(){}{\renewcommand\given{\delimsize\vert\mathopen{}}#1}

\newcommand\Photok{\ensuremath{\phi_k}\xspace}
\newcommand\fPhotok{\funop{\Photok}\paren}

\newcommand\VarPhoto{\ensuremath{\varphi}\xspace}
% \newcommand\fVarPhoto{\funop{\VarPhoto}\paren}

\DeclarePairedDelimiterXPP{\fVarPhoto}[1]{\funop{\VarPhoto}}(){}{\renewcommand\given{\delimsize\vert\mathopen{}}#1}

\newcommand\VarPhotok{\ensuremath{\varphi_k}\xspace}
\newcommand\fVarPhotok{\funop{\VarPhoto_k}\paren}


\newcommand\PhotoE[1]{\ensuremath{\phi_{\mathrm{E}_{#1}}}\xspace}
\newcommand\fPhotoE[1]{\funop{\PhotoE{#1}}\paren}

\newcommand\VarPhotoE[1]{\ensuremath{\varphi_{\mathrm{E}_{#1}}}\xspace}
\newcommand\fVarPhotoE[1]{\funop{\VarPhotoE{#1}}\paren}




\newcommand*\PhotoM{\ensuremath{\Photo_{\!\Manifold}}\xspace}
\newcommand\fPhotoM{\funop{\PhotoM}\paren}

\newcommand*\PhotoMt{\ensuremath{\Photo_{\!\Manifoldt}}\xspace}
\newcommand\fPhotoMt{\funop{\PhotoMt}\paren}


\newcommand*\Param{\ensuremath{\vec{\sigma}}\xspace}
\newcommand\fParam{\funop{\Param}\paren}

\newcommand*\Paramt{\ensuremath{\vec{\sigma}_{\!t}}\xspace}
\newcommand\fParamt{\funop{\Paramt}\paren}


\newcommand*\dirac{\delta}
\newcommand{\fdirac}[1]{\funop{\delta_{#1}}\paren}

\newcommand{\tangent}[1]{\ensuremath{\vec{\theta}_{#1}}\xspace}
\newcommand{\ftangent}[1]{\vec{\theta}_{#1}\paren}


\newcommand{\1}[1]{\funop{\indicator_{\!#1}}\paren}

\newcommand\unitInterval{\ensuremath{\intervalcc{0}{1}}\xspace}

\newcommand*\metric{\ensuremath{g}\xspace}
\newcommand*\metrict{\ensuremath{\metric_t}\xspace}
\newcommand{\fmetric}[1]{\funop{\metric_{#1}}\paren}
\newcommand{\fmetrict}{\funop{\metric_{t}}\paren}

\newcommand*\I{\ensuremath{I}\xspace}
\newcommand*\It{\ensuremath{\I_t}\xspace}
\newcommand{\fI}[1]{\funop{\I_{#1}}\paren}
\newcommand{\fIt}{\funop{\It}\paren}

\newcommand\jacobian{\ensuremath{\mat{J}}\xspace}
\newcommand\fjacobiant{\funop{\jacobian_{\!\Paramt}}\paren}

\newcommand*\measure{\ensuremath{\mu}\xspace}
%\newcommand*\fmeasure{\funop{\mu}\paren}
%\DeclarePairedDelimiterXPP{\fmeasure}[1]{\funop{\measure}}(){}{\renewcommand\given{\delimsize\vert\mathopen{}}#1}

\newcommand\f{\funop{f}\paren}

\newcommand\wquad{\ensuremath{w^{\mathrm{quad}}}\xspace}
\newcommand\fwquad{\funop{\wquad}\paren}

\newcommand\wgeom{\ensuremath{w^{\mathrm{geom}}}\xspace}
\newcommand\fwgeom{\funop{\wgeom}\paren}

\newcommand\VS{\ensuremath{\mathcal{VS}}\xspace}
\newcommand\RS{\ensuremath{\mathcal{RS}}\xspace}

\newcommand\w{\ensuremath{w}\xspace}
\newcommand\fw{\funop{w}\paren}

\newcommand\Npixel{\ensuremath{N_\mathrm{pixel}}\xspace}
\newcommand\Nsource{\ensuremath{N_\mathrm{source}}\xspace}




%-- Theorems
\newtheorem*{proposition}{Proposition}

\theoremstyle{remark}
\newtheorem*{example1}{Example 1}
\newtheorem*{example2}{Example 2}

\theoremstyle{theorem}
\newtheorem*{thm}{Theorem}

%-- Examples
% \[\funop{\left(\PSF \conv_\Manifold \Photo\right)}\paren{\vx} \defeq \int_\Manifold
% \fPSF{\vx-\vy} \fPhoto{\vy} \dvy \]

% \[\int_\Surf
% \fPSF{\vx-\vy} \fPhoto{\vy} \dvy = \int_\ParaDomain \fPSF{\vx-\fParam{u,v}}
% \fPhoto{u,v} \norm{\Tu \crossprod \Tv} \du \dv\]


% \[\left(K *_\mathcal{S} \phi\right)(\boldsymbol{x}) \defeq \int K(\boldsymbol{x}-\boldsymbol{\sigma}(u,v))
% \phi(u,v) \norm{\boldsymbol{T}_u \times \boldsymbol{T}_v} du dv \]


%%% Local Variables: 
%%% mode: latex
%%% TeX-master: "vm"
%%% End: 
